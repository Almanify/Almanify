\section{Einleitung}

Auf den folgenden Seiten befindet sich der ausführliche Projektbericht zur App \enquote{Almanify}.
In diesem Bericht werden die Anforderungen, die Entwicklungsprozesse und die Ergebnisse unserer Arbeit detailliert beschrieben.
Es wird auch auf die Herausforderungen eingegangen, mit denen das Entwicklungsteam während der Entwicklung konfrontiert wurden, und wie sie sie gelöst haben.
Zudem werden die Funktionen und Möglichkeiten der App aufgezeigt und eine Übersicht über die Ergebnisse der Entwicklung und der Benutzertests präsentiert.

\subsection{Problem und Zielsetzung}

Auf einer Reise mit mehreren Beteiligten kommt es oft vor, dass sich Teilnehmer gegenseitig Geld vorlegen, um eine Bezahlung zu vereinfachen.
Schnell kann es dabei unübersichtlich werden, insbesondere bei einem Roadtrip mit mehreren Freunden,
bei dem jeder für etwas anderes bezahlt und man nicht immer eine klare Übersicht hat.
Einer kauft noch schnell ein paar Snacks für alle, der andere fährt zwischendurch das Auto tanken,
und jemand hat kein Bargeld, um dem Fremdenführer Trinkgeld zu geben.
Daher muss er sich das Geld leihen.
Die App \enquote{Almanify} soll dabei helfen, diese komplexe Situation zu vereinfachen, indem alle Beteiligten die Ausgaben transparent miteinander verwalten können.
Am Ende sowie während der Reise soll es den Nutzern möglich sein, offenen Schulden gegenüber Mitreisenden einzusehen, um diese begleichen zu können.

\subsection{Anforderungen}

Bei Öffnung der App müssen sich Benutzer zunächst einloggen oder registrieren.
Dann können Sie Reisen erstellen und andere Nutzer über einen Invitecode oder QR-Code einladen.
Ausgaben im Rahmen einer Reise können hinzugefügt werden.
Beim Erstellen von Ausgaben gibt vorgegebene Auswahlmöglichkeiten für Währung, Kategorie und Beteiligte.
Nutzer können Einträge bearbeiten und Schulden gegenüber anderen auswerten.
Die App schlägt einen effizienten finanziellen Ausgleich vor und Schulden werden in eine vom User gewählte Währung umgerechnet.
Nach Abschluss einer Reise kann dies archiviert werden.

\subsection{Ausgewählte Technologien}

Die Anwendung basiert auf Angular.
Als Framework für die Implementierung der mobilen Anwendung kommt Ionic zum Einsatz.
Firebase dient als Backend für die Datenhaltung und Nutzerverwaltung.

Wir beschränken uns bei der Implementierung der nativen Komponenten der App auf die Android-Plattform.

\pagebreak

\subsection{Organisation und Vorgehensmodell}

Die Gruppe wurde zunächst mit einem ausführlichen Kickoff-Meeting, welches schon erste Mockups enthielt, vom Ideengeber über das Konzept der App aufgeklärt.
Im Anschluss an das Kickoff-Meeting wurden grundsätzliche Features und Designentscheidungen anhand der Mockups besprochen.

Im laufenden Projekt wurde sich an Scrum orientiert.
Die Sprints dauerten eine Woche.
Das Team traf sich jeden Mittwoch in Discord, um die Features der vorherigen Woche zu besprechen und zu sehen, wo Verbesserungen nötig waren.
Außerdem wurde diskutiert, welche Features in der kommenden Woche implementiert werden sollten.
