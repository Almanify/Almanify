\section{Anforderungen}

\subsection{Grundlegende Funktionsbeschreibung}

Bei Öffnung der App können sich Benutzer einloggen oder registrieren.
Dabei wird ein Benutzerprofil erstellt, das den Nutzernamen enthält.
Reisen können innerhalb der mobilen Anwendung erstellt werden.
Weitere Nutzer können über einen Invite-Code beitreten.
Getätigte Ausgaben können den beigetretenen Reisen hinzugefügt werden.
Einträge für Ausgaben umfassen einen Titel, die Person, die bezahlt hat, die Summe,
Währung, eine Kategorie, Zahldatum, die Beteiligten, die Ausgaben verursacht haben und ein optionales Bild (z. B. vom Kassenbon).
Für Währung, Kategorie und Beteiligte gibt es vorgegebene Auswahlmöglichkeiten.
Einträge für Ausgaben können entsprechend der Attribute sortiert werden.
Nutzer können Einträge für Ausgaben nachträglich bearbeiten.
Es ist möglich, Schulden bzw. Ansprüche gegenüber anderen auszuwerten.
Die App schlägt einen effizienten finanziellen Ausgleich unter den Beteiligten vor, sodass möglichst wenige Transaktionen nötig sind.
Schulden des Nutzers gegenüber anderen werden in eine von ihm angegebene Währung umgerechnet.
Nach Abschluss einer Reise kann dies archiviert werden.

\subsection{Funktionale Anforderungen}
Nun werden die funktionalen Anforderungen anhand von Use Cases beschrieben.

\setlist{nolistsep}

\begin{table}[H]
	\footnotesize
	\caption{Use Case: Einen Account erstellen}
	\begin{tabularx}{0.95\textwidth}{ |l|X| }
		\hline
		\rowcolor{gray} \textbf{Abschnitt}     & \textbf{Inhalt}                                                                                                                                       \\
		\hline
		Primärer Akteur                        & Nutzer                                                                                                                                                \\
		\hline
		Weitere Akteure                        & (Keine)                                                                                                                                               \\
		\hline
		Auslösende Ereignisse                  & Ein Nutzer öffnet die App und hat noch keinen Account. Er möchte jetzt einen Account erstellen.                                                       \\
		\hline
		\rowcolor{lightgray} \textbf{Szenario} & \textbf{Beschreibung}                                                                                                                                 \\
		\hline
		Hauptszenario                          & \begin{enumerate}[noitemsep]
			                                         \item Der Nutzer öffnet die App.
			                                         \item Die App zeigt die Anmeldeseite an.
			                                         \item Der Nutzer wählt auf der Anmeldeseite \enquote{Sign up} aus.
			                                         \item Er füllt dann seinen Namen, seine E-Mail-Adresse, und sein Passwort ein.
			                                         \item Er bestätigt sein Passwort durch erneute Eingabe.
			                                         \item Der Nutzer tippt auf \enquote{Sign up}, um seinen Account zu erstellen.
			                                         \item Die App zeigt erneut die Anmeldeseite mit nun vorausgefüllten Feldern an.
			                                         \item Es erscheint eine Meldung, dass der Account erfolgreich erstellt wurde.
		                                         \end{enumerate}                                                                        \\
		\hline
		Alternativszenarien                    & (Keine)                                                                                                                                               \\
		\hline
		Ausnahmeszenarien                      & \begin{enumerate}[noitemsep]
			                                         \item[4a.] Der Nutzer gibt keine gültige E-Mail-Adresse ein.
				                                         \subitem[4a1.] Die App zeigt eine Fehlermeldung an und fordert den Nutzer auf, eine gültige E-Mail-Adresse einzugeben.
				                                         \subitem[4a2.] Der \emph{Sign up}-Knopf ist ausgegraut, bis der Nutzer eine gültige E-Mail-Adresse eingegeben hat. \emph{Zurück zu 4.}
			                                         \item[4b.] Der Nutzer gibt ein Passwort ein, das nicht den Anforderungen entspricht.
				                                         \subitem[4b1.] Die App zeigt eine Fehlermeldung an und fordert den Nutzer auf, ein Passwort einzugeben, das die Anforderungen erfüllt.
				                                         \subitem[4b2.] Der \emph{Sign up}-Knopf ist ausgegraut, bis der Nutzer ein Passwort eingegeben hat, das die Anforderungen erfüllt. \emph{Zurück zu 4.}
			                                         \item[5a.] Der Nutzer gibt bei der Bestätigung des Passworts ein anderes Passwort ein.
				                                         \subitem[5a1.] Die App zeigt eine Fehlermeldung an und fordert den Nutzer auf, das gleiche Passwort erneut einzugeben.
				                                         \subitem[5a2.] Der \emph{Sign up}-Knopf ist ausgegraut, bis der Nutzer das gleiche Passwort erneut eingegeben hat. \emph{Zurück zu 5.}
		                                         \end{enumerate} \\
		\hline
		\rowcolor{lightgray}                   &                                                                                                                                                       \\
		\hline
		Vor-/ Nachbedingungen                  & \begin{enumerate}[noitemsep]
			                                         \item[Vor1.] Der Nutzer ist derzeit nicht mit der \emph{Remember Me}-Einstellung angemeldet.
			                                         \item[Nach1.] Der Nutzer besitzt einen Account.
		                                         \end{enumerate}                                                           \\
		\hline
	\end{tabularx}
\end{table}


\begin{table}[H]
	\footnotesize
	\caption{Use Case: In Account einloggen}
	\begin{tabularx}{0.95\textwidth}{ |l|X| }
		\hline
		\rowcolor{gray} \textbf{Abschnitt}     & \textbf{Inhalt}                                                                                                                       \\
		\hline
		Primärer Akteur                        & Nutzer                                                                                                                                \\
		\hline
		Weitere Akteure                        & (Keine)                                                                                                                               \\
		\hline
		Auslösende Ereignisse                  & Ein Nutzer öffnet die App und hat bereits einen Account. Er möchte sich jetzt anmelden.                                               \\
		\hline
		\rowcolor{lightgray} \textbf{Szenario} & \textbf{Beschreibung}                                                                                                                 \\
		\hline
		Hauptszenario                          & \begin{enumerate}
			                                         \item Der Nutzer öffnet die App. Er ist nicht angemeldet.
			                                         \item Die App zeigt die Anmeldeseite an.
			                                         \item Der Nutzer gibt seine E-Mail-Adresse und sein Passwort ein.
			                                         \item Er tippt auf \emph{Login}.
			                                         \item Er wird in den Account eingeloggt und auf seine aktive Reise weitergeleitet.
		                                         \end{enumerate}                                                     \\
		\hline
		Alternativszenarien                    & \begin{enumerate}
			                                         \item[1a.] Der Nutzer hat bei der letzten Anmeldung die \emph{Remember Me}-Einstellung aktiviert.
				                                         \subitem[1a1.] Die Anmeldedaten liegen gespeichert vor. \emph{Weiter mit 5.}
			                                         \item[5a.] Der Nutzer hat keine aktive Reise.
				                                         \subitem[5a1.] Die App zeigt die Startseite an.
		                                         \end{enumerate}                                      \\
		\hline
		Ausnahmeszenarien                      & \begin{enumerate}[noitemsep]
			                                         \item[3a.] Der Nutzer gibt keine gültige E-Mail-Adresse ein.
				                                         \subitem[3a1.] Die App zeigt eine Fehlermeldung an und fordert den Nutzer auf, eine gültige E-Mail-Adresse einzugeben.
				                                         \subitem[3a2.] Der \emph{Login}-Knopf ist ausgegraut, bis der Nutzer eine gültige E-Mail-Adresse eingegeben hat. \emph{Zurück zu 3.}
			                                         \item[4a.] Die Zugangsdaten sind ungültig.
				                                         \subitem[4a1.] Die App zeigt eine Fehlermeldung an und fordert den Nutzer auf, die Zugangsdaten erneut einzugeben. \emph{Zurück zu 3.}
		                                         \end{enumerate} \\
		\hline
		\rowcolor{lightgray}                   &                                                                                                                                       \\
		\hline
		Vor-/ Nachbedingungen                  & \begin{enumerate}[noitemsep]
			                                         \item[Vor1.] Der Nutzer besitzt einen Account.
		                                         \end{enumerate}                                                                                         \\
		\hline
	\end{tabularx}
\end{table}


\begin{table}[H]
	\caption{Use Case: Eine Reise erstellen}
	\footnotesize
	\begin{tabularx}{0.95\textwidth}{ |l|X| }
		\hline
		\rowcolor{gray} \textbf{Abschnitt}     & \textbf{Inhalt}                                                                                                                     \\
		\hline
		Primärer Akteur                        & Reiseleiter                                                                                                                         \\
		\hline
		Weitere Akteure                        & (Keine)                                                                                                                             \\
		\hline
		Auslösende Ereignisse                  & Ein Nutzer möchte eine neue Reise anlegen.                                                                                          \\
		\hline
		\rowcolor{lightgray} \textbf{Szenario} & \textbf{Beschreibung}                                                                                                               \\
		\hline
		Hauptszenario                          & \begin{enumerate}
			                                         \item Der angemeldete Nutzer navigiert zur Startseite (\emph{Home}).
			                                         \item Er tippt auf \emph{Create Journey}.
			                                         \item Ein Dialog öffnet sich, in dem er die Reiseinformationen eingeben kann.
			                                         \item Er gibt einen Titel, eine Standardwährung sowie ein Start- und Enddatum ein.
			                                         \item Er tippt auf \emph{Save}.
			                                         \item Nach Erstellen der Reise öffnet sich das Einladungsmodal.
		                                         \end{enumerate}                                                   \\
		\hline
		Alternativszenarien                    & (Keine)                                                                                                                             \\
		\hline
		Ausnahmeszenarien                      & \begin{enumerate}
			                                         \item[4a.] Der Nutzer gibt einen ungültigen Reisezeitraum ein.
				                                         \subitem[4a1.] Die App zeigt eine Fehlermeldung an und fordert den Nutzer auf, einen gültigen Reisezeitraum einzugeben.
				                                         \subitem[4a2.] Der \emph{Save}-Knopf ist ausgegraut, bis der Nutzer einen gültigen Reisezeitraum eingegeben hat. \emph{Zurück zu 4.}
		                                         \end{enumerate} \\
		\hline
		\rowcolor{lightgray}                   &                                                                                                                                     \\
		\hline
		Vor-/ Nachbedingungen                  & \begin{enumerate}
			                                         \item[Vor1.] Der Nutzer besitzt einen Account und ist angemeldet.
			                                         \item[Nach1.] Es existiert eine neue Reise.
		                                         \end{enumerate}                                                                    \\
		\hline
	\end{tabularx}
\end{table}

\begin{table}[H]
	\footnotesize
	\caption{Use Case: Eine Reise beitreten}
	\begin{tabularx}{0.95\textwidth}{ |l|X| }
		\hline
		\rowcolor{gray} \textbf{Abschnitt}     & \textbf{Inhalt}                                                                  \\
		\hline
		Primärer Akteur                        & Nutzer                                                                           \\
		\hline
		Weitere Akteure                        & Reiseleiter                                                                      \\
		\hline
		Auslösende Ereignisse                  & Der Nutzer möchte einer Reise beitreten, die der Reiseleiter erstellt hat.       \\
		\hline
		\rowcolor{lightgray} \textbf{Szenario} & \textbf{Beschreibung}                                                            \\
		\hline
		Hauptszenario                          & \begin{enumerate}
			                                         \item Der angemeldete Nutzer navigiert zur Startseite (\emph{Home}).
			                                         \item Er tippt auf \emph{Join Journey}.
			                                         \item Er gibt den Invitecode ein, den der Reiseleiter ihm mitgeteilt hat.
			                                         \item Er tippt auf \emph{Apply}.
			                                         \item Der Nutzer wird zur Reise hinzugefügt. Die App leitet ihn zur Reise weiter.
		                                         \end{enumerate} \\
		\hline
		Alternativszenarien                    & \begin{enumerate}
			                                         \item[3a.] Der Nutzer benutzt das QR-Code-Feature.
				                                         \subitem[3a1.] Der Nutzer tippt auf \emph{Scan QR-Code}.
				                                         \subitem[3a2.] Die App öffnet die Kamera.
				                                         \subitem[3a3.] Der Nutzer scannt den QR-Code. \emph{Weiter mit 5.}
		                                         \end{enumerate}                \\
		\hline
		Ausnahmeszenarien                      & \begin{enumerate}
			                                         \item[3b.] Der Nutzer gibt einen ungültigen Invitecode ein.
				                                         \subitem[3b1.] Die App meldet einen invaliden Invitecode. \emph{Zurück zu 3.}
		                                         \end{enumerate}     \\
		\hline
		\rowcolor{lightgray}                   &                                                                                  \\
		\hline
		Vor-/ Nachbedingungen                  & \begin{enumerate}
			                                         \item[Vor1.] Der Nutzer besitzt einen Account und ist angemeldet.
			                                         \item[Vor2.] Der Nutzer ist nicht Mitglied der Reise.
			                                         \item[Nach1.] Der Nutzer ist Mitglied der Reise.
		                                         \end{enumerate}                 \\
		\hline
	\end{tabularx}
\end{table}


\begin{table}[H]
	\footnotesize
	\caption{Use Case: Zahlung festhalten}
	\begin{tabularx}{0.95\textwidth}{ |l|X| }
		\hline
		\rowcolor{gray} \textbf{Abschnitt}     & \textbf{Inhalt}                                                                                                              \\
		\hline
		Primärer Akteur                        & Reisemitglied                                                                                                                \\
		\hline
		Weitere Akteure                        & (Keine)                                                                                                                      \\
		\hline
		Auslösende Ereignisse                  & Ein Reisemitglied möchte eine neue Zahlung festhalten, die er oder eine andere Person im Rahmen der Reise getätigt hat.      \\
		\hline
		\rowcolor{lightgray} \textbf{Szenario} & \textbf{Beschreibung}                                                                                                        \\
		\hline
		Hauptszenario                          & \begin{enumerate}
			                                         \item Das Reisemitglied ist auf der Seite der Reise.
			                                         \item Er tippt auf \emph{New Payment}.
			                                         \item Ein Dialog öffnet sich, in dem er die Daten der Zahlung eingibt.
			                                         \item Er fügt einen Titel hinzu, wählt den Zahlenden, die Währung, den Betrag, das Datum und die Kategorie aus.
			                                         \item Er wählt die Reiseteilnehmer aus, für die die Zahlung getätigt wurde.
			                                         \item Er tippt auf \emph{Save}. Die Zahlung wird gespeichert.
		                                         \end{enumerate}               \\
		\hline
		Alternativszenarien                    & \begin{enumerate}
			                                         \item[4a.] Der Nutzer möchte zusätzlich ein Bild einfügen.
				                                         \subitem[4a1.] Er tippt auf die \emph{Add Image} Schaltfläche.
				                                         \subitem[4a2.] Die App öffnet die Kamera.
				                                         \subitem[4a3.] Der Nutzer macht ein Foto, welches hochgeladen wird. \emph{Zurück zu 4.}
		                                         \end{enumerate}                                       \\
		\hline
		Ausnahmeszenarien                      & \begin{enumerate}
			                                         \item[4b.] Der Nutzer gibt einen ungültigen (negativen) Betrag ein.
				                                         \subitem[4b1.] Die App meldet einen invaliden Betrag.
				                                         \subitem[4a2.] Der \emph{Save}-Knopf ist ausgegraut, bis der Nutzer einen gültigen Betrag eingegeben hat. \emph{Zurück zu 4.}
		                                         \end{enumerate} \\
		\hline
		\rowcolor{lightgray}                   &                                                                                                                              \\
		\hline
		Vor-/ Nachbedingungen                  & \begin{enumerate}
			                                         \item[Nach1.] Die Zahlung ist gespeichert.
		                                         \end{enumerate}                                                                                    \\
		\hline
	\end{tabularx}
\end{table}


\begin{table}[H]
	\footnotesize
	\caption{Use Case: Zahlung einsehen}
	\begin{tabularx}{0.95\textwidth}{ |l|X| }
		\hline
		\rowcolor{gray} \textbf{Abschnitt}     & \textbf{Inhalt}                                                                                    \\
		\hline
		Primärer Akteur                        & Reisemitglied                                                                                      \\
		\hline
		Weitere Akteure                        & (Keine)                                                                                            \\
		\hline
		Auslösende Ereignisse                  & Ein Reisemitglied möchte eine eingetragene Zahlung einsehen.                                       \\
		\hline
		\rowcolor{lightgray} \textbf{Szenario} & \textbf{Beschreibung}                                                                              \\
		\hline
		Hauptszenario                          & \begin{enumerate}
			                                         \item Das Reisemitglied ist auf der Seite der Reise.
			                                         \item Er tippt auf eine Zahlung.
			                                         \item Die Zahlung weitet sich aus und zeigt grundlegende Informationen wie Zahler und Zahldatum an.
			                                         \item Der Nutzer tippt auf das Auge, um die Details der Zahlung einzusehen.
			                                         \item Die Zahlungsdetails werden angezeigt.
		                                         \end{enumerate} \\
		\hline
		Alternativszenarien                    & (Keine)                                                                                            \\
		\hline
		Ausnahmeszenarien                      & (Keine)                                                                                            \\
		\hline
		\rowcolor{lightgray}                   &                                                                                                    \\
		\hline
		Vor-/ Nachbedingungen                  & \begin{enumerate}
			                                         \item[Vor1.] Es existiert eine Zahlung.
		                                         \end{enumerate}                                                             \\
		\hline
	\end{tabularx}
\end{table}


\begin{table}[H]
	\footnotesize
	\caption{Use Case: Schulden einsehen}
	\begin{tabularx}{0.95\textwidth}{ |l|X| }
		\hline
		\rowcolor{gray} \textbf{Abschnitt}     & \textbf{Inhalt}                                                                                                                        \\
		\hline
		Primärer Akteur                        & Reisemitglied                                                                                                                          \\
		\hline
		Weitere Akteure                        & (Keine)                                                                                                                                \\
		\hline
		Auslösende Ereignisse                  & Ein Reisemitglied möchte einsehen, welche Schulden er hat bzw. was ihm geschuldet wird.                                                \\

		\hline
		\rowcolor{lightgray} \textbf{Szenario} & \textbf{Beschreibung}                                                                                                                  \\
		\hline
		Hauptszenario                          & \begin{enumerate}
			                                         \item Das Reisemitglied ist auf der Seite der Reise.
			                                         \item Er tippt auf \emph{Debts}.
			                                         \item Der Schuldenrechner öffnet sich.
			                                         \item Dem Nutzer wird die Gesamtsumme an Geld angezeigt, das er anderen Reisemitgliedern schuldet.
			                                         \item Darunter werden die einzelnen Schulden aufgeführt.
			                                         \item Der Nutzer kann die Währung ändern, indem er auf \emph{Currency} tippt.
			                                         \item Der Nutzer kann eine getätigte Ausgleichszahlung eintragen, indem er auf \emph{Pay} neben einem Betrag tippt.
		                                         \end{enumerate}                     \\
		\hline
		Alternativszenarien                    & \begin{enumerate}
			                                         \item[4a.] Der Nutzer hat keine Schulden, aber andere Reisemitglieder schulden ihm Geld.
				                                         \subitem[4a1.] Dem Nutzer wird die Gesamtsumme an Geld angezeigt, die andere Reisemitglieder ihm schulden.
				                                         \subitem[4a2.] Darunter werden die einzelnen Schulden aufgeführt.
				                                         \subitem[4a3.] Der Nutzer kann die Währung ändern, indem er auf \emph{Currency} tippt.
				                                         \subitem[4a4.] Der Nutzer kann eine getätigte Ausgleichszahlung eintragen, indem er auf \emph{Mark as repaid} neben einem Betrag tippt.
			                                         \item[4b.] Der Nutzer hat keine Schulden, und andere Reisemitglieder schulden ihm auch nichts.
				                                         \subitem[4b1.] Dem Nutzer wird angezeigt, dass er keine Schulden hat.
		                                         \end{enumerate} \\
		\hline
		Ausnahmeszenarien                      & (Keine)                                                                                                                                \\
		\hline
		\rowcolor{lightgray}                   &                                                                                                                                        \\
		\hline
		Vor-/ Nachbedingungen                  & (Keine)                                                                                                                                \\
		\hline
	\end{tabularx}
\end{table}


\begin{table}[H]
	\footnotesize
	\caption{Use Case: Reise archivieren}
	\begin{tabularx}{0.95\textwidth}{ |l|X| }
		\hline
		\rowcolor{gray} \textbf{Abschnitt}     & \textbf{Inhalt}                                              \\
		\hline
		Primärer Akteur                        & Reiseleiter                                                  \\
		\hline
		Weitere Akteure                        & (Keine)                                                      \\
		\hline
		Auslösende Ereignisse                  & Ein Reiseleiter möchte eine Reise archivieren.               \\

		\hline
		\rowcolor{lightgray} \textbf{Szenario} & \textbf{Beschreibung}                                        \\
		\hline
		Hauptszenario                          & \begin{enumerate}
			                                         \item Der Reiseleiter ist auf der Liste aller Reisen.
			                                         \item Er scrollt zur seiner Reise, die er archivieren möchte.
			                                         \item Er tippt auf das Schlosssymbol.
			                                         \item Er bestätigt die Archivierung.
		                                         \end{enumerate} \\
		\hline
		Alternativszenarien                    & (Keine)                                                      \\
		\hline
		Ausnahmeszenarien                      & (Keine)                                                      \\
		\hline
		\rowcolor{lightgray}                   &                                                              \\
		\hline
		Vor-/ Nachbedingungen                  & \begin{enumerate}
			                                         \item[Vor1.] Die Reise darf nicht bereits archiviert sein.
			                                         \item[Nach1.] Die Reise ist archiviert.
		                                         \end{enumerate}    \\
		\hline
	\end{tabularx}
\end{table}


\begin{table}[H]
	\footnotesize
	\caption{Use Case: Reisen einsehen}
	\begin{tabularx}{0.95\textwidth}{ |l|X| }
		\hline
		\rowcolor{gray} \textbf{Abschnitt}     & \textbf{Inhalt}                                                  \\
		\hline
		Primärer Akteur                        & Reisemitglied                                                    \\
		\hline
		Weitere Akteure                        & (Keine)                                                          \\
		\hline
		Auslösende Ereignisse                  & Das Reisemitglied möchte eine Reise mit deren Zahlungen einsehen \\

		\hline
		\rowcolor{lightgray} \textbf{Szenario} & \textbf{Beschreibung}                                            \\
		\hline
		Hauptszenario                          & \begin{enumerate}
			                                         \item Das Reisemitglied ist auf der Liste aller Reisen.
			                                         \item Er scrollt zur Reise, die er einsehen möchte.
			                                         \item Er tippt auf das Bild der Reise.
			                                         \item Er wird auf die Reise weitergeleitet.
		                                         \end{enumerate}           \\
		\hline
		Alternativszenarien                    & (Keine)                                                          \\
		\hline
		Ausnahmeszenarien                      & (Keine)                                                          \\
		\hline
		\rowcolor{lightgray}                   &                                                                  \\
		\hline
		Vor-/ Nachbedingungen                  & \begin{enumerate}
			                                         \item[Nach1.] Das Reisemitglied ist auf der Seite der Reise.
		                                         \end{enumerate}      \\
		\hline
	\end{tabularx}
\end{table}

\subsection{Nichtfunktionale Anforderungen}

\begin{enumerate}
	\item \textbf{Sicherheit:} Die App soll sicher sein, d. h. keine sensiblen Daten wie Passwörter oder Kreditkartendaten speichern.\\
	So soll die Menge an Daten, die von Nutzern in der App oder auf dem Server gespeichert werden, möglichst klein gehalten werden.
	\item \textbf{Benutzerfreundlichkeit:} Die App soll einfach zu bedienen sein, damit die Nutzer schnell und einfach mit ihr arbeiten können.\\
	Dazu soll die App eine intuitive Benutzeroberfläche haben, die den Nutzern die Bedienung der App erleichtert.
	\item \textbf{Schnelligkeit:} Die App soll schnell reagieren, damit die Nutzer nicht lange auf die Ergebnisse warten müssen.
	\item \textbf{Robustheit:} Die App soll robust sein, d. h. sie soll auch bei Fehlern oder unerwarteten Eingaben nicht abstürzen.\\
	Dazu soll die App Fehlermeldungen ausgeben, wenn sie nicht mit den Eingaben umgehen kann, und in einen sicheren Zustand zurückkehren.
	\item \textbf{Wartbarkeit:} Die App soll einfach zu warten sein, damit die Entwickler schnell und einfach Fehler beheben können.\\
	Dazu soll die App gut dokumentiert sein und eine gute Struktur haben.
	Code soll sinnvoll aufgeteilt sein und in einzelne Funktionen ausgelagert werden.
	Wenn möglich soll der Code wiederverwendet werden, um Redundanzen zu vermeiden.
	\item \textbf{Portierbarkeit:} Die App soll auf möglichst vielen Geräten laufen, damit möglichst viele Nutzer sie verwenden können.\\
	Dazu soll die App auf Android und iOS laufen und das Design der App an die jeweiligen Geräte angepasst sein.
	\item \textbf{Korrektheit:} Die App soll den funktionalen Anforderungen genügen.
	\item \textbf{Zuverlässigkeit:} Die App soll den Vorstellungen der Nutzer entsprechen.
\end{enumerate}

\subsection{(Benutzer) Schnittstellen / Ein-Ausgabeformate}

Die App soll mit dem Server kommunizieren, um die Daten zu erhalten und zu speichern.
Es gibt festgelegte Datenstrukturen zu Nutzern, Reisen und Zahlungen, die der Server und die App verwenden.

\begin{itemize}
	\item \textbf{Nutzer:} Ein Nutzer hat einen Namen und eine Standardwährung\footnote{Die E-Mail-Adresse und das Passwort wird von Firebase verwaltet.}.
	\item \textbf{Reise:} Eine Reise hat einen Namen, eine Beschreibung, ein Startdatum, ein Enddatum, eine Standardwährung, einen Invite-Code, einen Reiseleiter und eine Liste von Reisemitgliedern.
	\item \textbf{Zahlung:} Eine Zahlung hat eine zugeordnete Reise, einen Betrag, einen Absender, einen Empfänger, eine Beschreibung, ein Datum und eine Währung und ein optionales Bild.
\end{itemize}

\subsection{Fehlverhalten}

Werden invalide Daten eingegeben, soll die App eine Fehlermeldung ausgeben und in einen sicheren Zustand zurückkehren.
Dies soll ebenfalls für zu große Datenmengen gelten.
Um solche Fehler zu vermeiden, soll die App die Eingaben validieren.

\subsection{Abnahmekriterien}

Die App soll die nicht- und funktionalen Anforderungen erfüllen.
Dabei haben die funktionalen Anforderungen Priorität.
Die App soll die Datenstrukturen (ggf. nach Bedarf angepasst) verwenden, um mit dem Server zu kommunizieren.
Sie soll die Daten validieren und sicher Fehlermeldungen ausgeben, wenn die Eingaben nicht valide sind.
