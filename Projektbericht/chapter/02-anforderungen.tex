\section{Anforderungen}

Bei Öffnung der App können sich Benutzer einloggen oder registrieren.
Dabei wird ein Benutzerprofil erstellt, das den Nutzernamen enthält.
Reisen können innerhalb der mobilen Anwendung erstellt werden.
Weitere Nutzer können über einen Invitecode oder QR-Code beitreten.
Getätigte Ausgaben können den beigetretenen Reisen hinzugefügt werden.
Einträge für Ausgaben umfassen einen Titel, die Person, die bezahlt hat, die Summe,
Währung, eine Kategorie, Zahldatum, die Beteiligten, die Ausgaben verursacht haben,
eine optionale Markierung auf Google Maps wo die Zahlung getätigt wurde und ein optionales Bild vom Kassenbon.
Für Währung, Kategorie und Beteiligte gibt es vorgegebene Auswahlmöglichkeiten.
Einträge für Ausgaben können entsprechend der Attribute sortiert werden.
Nutzer können Einträge für Ausgaben nachträglich bearbeiten.
Es ist möglich, Schulden bzw. Ansprüche gegenüber anderen auszuwerten.
Die App schlägt einen effizienten finanziellen Ausgleich unter den Beteiligten vor, sodass möglichst wenige Transaktionen nötig sind.
Schulden des Nutzers gegenüber anderen werden in eine von ihm angegebene Währung umgerechnet.
Nach Abschluss einer Reise kann dies archiviert werden.

\subsection{Grundlegende Funktionsbeschreibung}

Funktionsbeschreibung anhand von User Stories:

\begin{table}[H]
	\caption{Einen Account erstellen}
	\begin{tabularx}{0.95\textwidth}{ |X|X| }
		\hline
		\rowcolor{gray} \textbf{Abschnitt}     & \textbf{Inhalt}       \\
		\hline
		Primärer Akteur (Initiator)            & Nutzer                \\
		\hline
		Weitere Akteure                        & -                     \\
		\hline
		Auslösende Ereignisse (Trigger)        & [--TODO--]            \\
		\hline
		\rowcolor{lightgray} \textbf{Szenario} & \textbf{Beschreibung} \\
		\hline
		Hauptszenario                          & [--TODO--]            \\
		\hline
		Alternativszenarien                    & [--TODO--]            \\
		\hline
		Ausnahmeszenarien                      & [--TODO--]            \\
		\hline
		\rowcolor{lightgray}                   &                       \\
		\hline
		Vor-/ Nachbedingungen                  & [--TODO--]            \\
		\hline
	\end{tabularx}
\end{table}


\begin{table}[H]
	\caption{In Account einloggen}
	\begin{tabularx}{0.95\textwidth}{ |X|X| }
		\hline
		\rowcolor{gray} \textbf{Abschnitt}     & \textbf{Inhalt}                      \\
		\hline
		Primärer Akteur (Initiator)            & Nutzer                               \\
		\hline
		Weitere Akteure                        & -                                    \\
		\hline
		Auslösende Ereignisse (Trigger)        & [--TODO--]                           \\
		\hline
		\rowcolor{lightgray} \textbf{Szenario} & \textbf{Beschreibung}                \\
		\hline
		Hauptszenario                          & [--TODO--] [Beitritt via Invitecode] \\
		\hline
		Alternativszenarien                    & [--TODO--] [Beitritt via QR-Code]    \\
		\hline
		Ausnahmeszenarien                      & [--TODO--]                           \\
		\hline
		\rowcolor{lightgray}                   &                                      \\
		\hline
		Vor-/ Nachbedingungen                  & [--TODO--]                           \\
		\hline
	\end{tabularx}
\end{table}


\begin{table}[H]
	\caption{Eine Reise erstellen/beitreten}
	\begin{tabularx}{0.95\textwidth}{ |X|X| }
		\hline
		\rowcolor{gray} \textbf{Abschnitt}     & \textbf{Inhalt}                           \\
		\hline
		Primärer Akteur (Initiator)            & Reiseleiter                               \\
		\hline
		Weitere Akteure                        & Reisemitglied                             \\
		\hline
		Auslösende Ereignisse (Trigger)        & [--TODO--]                                \\
		\hline
		\rowcolor{lightgray} \textbf{Szenario} & \textbf{Beschreibung}                     \\
		\hline
		Hauptszenario                          & [--TODO--] [normales Einloggen]           \\
		\hline
		Alternativszenarien                    & [--TODO--] [\emph{Remember Me} Einloggen] \\
		\hline
		Ausnahmeszenarien                      & [--TODO--]                                \\
		\hline
		\rowcolor{lightgray}                   &                                           \\
		\hline
		Vor-/ Nachbedingungen                  & [--TODO--]                                \\
		\hline
	\end{tabularx}
\end{table}


\begin{table}[H]
	\caption{Zahlung festhalten}
	\begin{tabularx}{0.95\textwidth}{ |X|X| }
		\hline
		\rowcolor{gray} \textbf{Abschnitt}     & \textbf{Inhalt}       \\
		\hline
		Primärer Akteur (Initiator)            & Reisemitglied         \\
		\hline
		Weitere Akteure                        & -                     \\
		\hline
		Auslösende Ereignisse (Trigger)        & [--TODO--]            \\
		\hline
		\rowcolor{lightgray} \textbf{Szenario} & \textbf{Beschreibung} \\
		\hline
		Hauptszenario                          & [--TODO--]            \\
		\hline
		Alternativszenarien                    & [--TODO--]            \\
		\hline
		Ausnahmeszenarien                      & [--TODO--]            \\
		\hline
		\rowcolor{lightgray}                   &                       \\
		\hline
		Vor-/ Nachbedingungen                  & [--TODO--]            \\
		\hline
	\end{tabularx}
\end{table}


\begin{table}[H]
	\caption{Zahlung einsehen}
	\begin{tabularx}{0.95\textwidth}{ |X|X| }
		\hline
		\rowcolor{gray} \textbf{Abschnitt}     & \textbf{Inhalt}       \\
		\hline
		Primärer Akteur (Initiator)            & Reisemitglied         \\
		\hline
		Weitere Akteure                        & -                     \\
		\hline
		Auslösende Ereignisse (Trigger)        & [--TODO--]            \\
		\hline
		\rowcolor{lightgray} \textbf{Szenario} & \textbf{Beschreibung} \\
		\hline
		Hauptszenario                          & [--TODO--]            \\
		\hline
		Alternativszenarien                    & [--TODO--]            \\
		\hline
		Ausnahmeszenarien                      & [--TODO--]            \\
		\hline
		\rowcolor{lightgray}                   &                       \\
		\hline
		Vor-/ Nachbedingungen                  & [--TODO--]            \\
		\hline
	\end{tabularx}
\end{table}


\begin{table}[H]
	\caption{Schulden einsehen}
	\begin{tabularx}{0.95\textwidth}{ |X|X| }
		\hline
		\rowcolor{gray} \textbf{Abschnitt}     & \textbf{Inhalt}       \\
		\hline
		Primärer Akteur (Initiator)            & Reisemitglied         \\
		\hline
		Weitere Akteure                        & -                     \\
		\hline
		Auslösende Ereignisse (Trigger)        & [--TODO--]            \\

		\hline
		\rowcolor{lightgray} \textbf{Szenario} & \textbf{Beschreibung} \\
		\hline
		Hauptszenario                          & [--TODO--]            \\
		\hline
		Alternativszenarien                    & [--TODO--]            \\
		\hline
		Ausnahmeszenarien                      & [--TODO--]            \\
		\hline
		\rowcolor{lightgray}                   &                       \\
		\hline
		Vor-/ Nachbedingungen                  & [--TODO--]            \\
		\hline
	\end{tabularx}
\end{table}


\begin{table}[H]
	\caption{Reise archivieren}
	\begin{tabularx}{0.95\textwidth}{ |X|X| }
		\hline
		\rowcolor{gray} \textbf{Abschnitt}     & \textbf{Inhalt}       \\
		\hline
		Primärer Akteur (Initiator)            & Reiseleiter           \\
		\hline
		Weitere Akteure                        & -                     \\
		\hline
		Auslösende Ereignisse (Trigger)        & [--TODO--]            \\

		\hline
		\rowcolor{lightgray} \textbf{Szenario} & \textbf{Beschreibung} \\
		\hline
		Hauptszenario                          & [--TODO--]            \\
		\hline
		Alternativszenarien                    & [--TODO--]            \\
		\hline
		Ausnahmeszenarien                      & [--TODO--]            \\
		\hline
		\rowcolor{lightgray}                   &                       \\
		\hline
		Vor-/ Nachbedingungen                  & [--TODO--]            \\
		\hline
	\end{tabularx}
\end{table}


\begin{table}[H]
	\caption{Reisen einsehen}
	\begin{tabularx}{0.95\textwidth}{ |X|X| }
		\hline
		\rowcolor{gray} \textbf{Abschnitt}     & \textbf{Inhalt}       \\
		\hline
		Primärer Akteur (Initiator)            & Reisemitglied         \\
		\hline
		Weitere Akteure                        & -                     \\
		\hline
		Auslösende Ereignisse (Trigger)        & [--TODO--]            \\

		\hline
		\rowcolor{lightgray} \textbf{Szenario} & \textbf{Beschreibung} \\
		\hline
		Hauptszenario                          & [--TODO--]            \\
		\hline
		Alternativszenarien                    & [--TODO--]            \\
		\hline
		Ausnahmeszenarien                      & [--TODO--]            \\
		\hline
		\rowcolor{lightgray}                   &                       \\
		\hline
		Vor-/ Nachbedingungen                  & [--TODO--]            \\
		\hline
	\end{tabularx}
\end{table}


\subsection{Funktionale Anforderungen}

\subsection{ Nicht funktionale Anforderungen}

\subsection{(Benutzer) Schnittstellen / Ein-Ausgabeformate}

\subsection{Fehlverhalten}

\subsection{Abnahmekriterien}